\documentclass[12pt,a4paper,titlepage,headinclude,bibtotoc]{scrartcl}

%---- Allgemeine Layout Einstellungen ------------------------------------------

% Für Kopf und Fußzeilen, siehe auch KOMA-Skript Doku
\usepackage[komastyle]{scrpage2}
\pagestyle{plain}
\setheadsepline{0.5pt}[\color{black}]
\automark[section]{chapter}


%Einstellungen für Figuren- und Tabellenbeschriftungen
\setkomafont{captionlabel}{\sffamily\bfseries}
\setcapindent{0em}


%---- Weitere Pakete -----------------------------------------------------------
% Die Pakete sind alle in der TeX Live Distribution enthalten. Wichtige Adressen
% www.ctan.org, www.dante.de

% Sprachunterstützung
\usepackage[ngerman]{babel}

% Benutzung von Umlauten direkt im Text
% entweder "latin1" oder "utf8"
\usepackage[utf8]{inputenc}

% Pakete mit Mathesymbolen und zur Beseitigung von Schwächen der Mathe-Umgebung
\usepackage{latexsym,exscale,stmaryrd,amssymb,amsmath}


\usepackage[nointegrals]{wasysym}
\usepackage{eurosym}

% Anderes Literaturverzeichnisformat
%\usepackage[square,sort&compress]{natbib}
\usepackage{hyperref}
% Für Farbe
\usepackage{color}
\usepackage{graphicx}
\usepackage{wrapfig}
\usepackage{subfigure}

% Caption neben Abbildung
\usepackage{sidecap}


% Befehl für "Entspricht"-Zeichen
\newcommand{\corresponds}{\ensuremath{\mathrel{\widehat{=}}}}
% Befehl für Errorfunction
\newcommand{\erf}[1]{\text{ erf}\ensuremath{\left( #1 \right)}}


%Fußnoten zwingend auf diese Seite setzen
\interfootnotelinepenalty=1000

%Für chemische Formeln (von www.dante.de)
%% Anpassung an LaTeX(2e) von Bernd Raichle
\makeatletter
\DeclareRobustCommand{\chemical}[1]{%
  {\(\m@th
   \edef\resetfontdimens{\noexpand\)%
       \fontdimen16\textfont2=\the\fontdimen16\textfont2
       \fontdimen17\textfont2=\the\fontdimen17\textfont2\relax}%
   \fontdimen16\textfont2=2.7pt \fontdimen17\textfont2=2.7pt
   \mathrm{#1}%
   \resetfontdimens}}
\makeatother
\usepackage{textcomp}
\usepackage{upgreek}
%\begin{document}
%$\upmu$
%\end{document}
%Honecker-Kasten mit $$\shadowbox{$xxxx$}$$
\usepackage{fancybox}

%SI-Package
\usepackage{siunitx}

%keine Einrückung, wenn Latex doppelte Leerzeile
\parindent0pt

%Bibliography \bibliography{literatur} und \cite{gerthsen}
%\usepackage{cite}
\usepackage{babelbib}
\selectbiblanguage{ngerman}

\usepackage{siunitx}
%\begin{document}
 % \SI{1.55}{\micro\metre}
\sisetup{math-micro=\text{µ},text-micro=µ}
\usepackage{amsmath}

\usepackage[verbose]{placeins}
%für \FloatBarrier

\begin{document}

\begin{titlepage}
\centering
\textsc{\Large Physikalisch- Chemisches Grundpraktikum\\[1.5ex] Universität Göttingen}

\vspace*{0.5cm}

\rule{\textwidth}{1pt}\\[0.5cm]
{\huge \bfseries
  Versuch 1: \\[1.5ex]
  Molare Wärmekapazität von Festkörpern }\\[0.5cm]
\rule{\textwidth}{1pt}

\vspace*{0.5cm}


\begin{Large}
\begin{tabular}{ll}
Durchführende: &  Isaac Maksso, Julia Stachowiak\\
Assistent: & Sven Meyer \\
 Versuchsdatum: & 10.11.2016\\
 Datum der ersten Abgabe: & 17.11.2017\\
\end{tabular}
\end{Large}

\vspace*{0.5cm}


\begin{table}[h!]
\centering
\caption{Ergebnisse des Versuchs.}
\begin{tabular}{c|c|c|c||c|c}
Probe&Temperaturbad&$\text{c}_P^{\text{Exp.}}$ [$\frac{\text{J}}{\text{mol}\cdot\text{K}}$] &$\text{c}_P^{\text{Lit.}}$ [$\frac{\text{J}}{\text{mol}\cdot\text{K}}$] & $<\Theta_D >$ [K]& $<\Theta_{D,Lit.}>$ [K] \\
\hline
Graphit& ZT&4,52 ± 0,053  &8,517& 138$ \cdot 10^2$& 2500950\\
\hline
Zink &ZT& 51,7 ± 0,260 &24,47& 981 & 345\\
\hline
Kupfer &ZT&55,9 ± 0,242 &25,330& 630 &308\\
\end{tabular}
\end{table}
\end{titlepage}


\tableofcontents

\newpage


\section{Experimentelles}
\subsection{Experimenteller Aufbau}

\subsection{Durchführung}
Es wurde 0.4002\;g Silberiodid abgewogen und zu einer Tablette gepresst. Es wurde eine Feststoffkette, wie in Abbildung ... zu sehen ist, aufgebaut und 10\;min mit $\text{N}_2$-Gas umspült. Nach einer Aufheizphase auf 160\;°C hochgeheizt und 45\;min bei einem Strom von 1.2\;mA aufgeladen. Es wurde ab 160\;°C in 5\;°C-Schritten die Spannung gemessen. Ab 175\;°C wurde das Messgerät kurzgeschlossen und die Messung fortgesetzt. 
\section{Auswertung}
\subsection{Messergebnisse}
In der Tabelle 2 sind die Messergebnisse der Elektromotorischenkraft dargestellt.
\begin{table}[h!]
\centering
\caption{Messergebnisse des Versuchs.}
\begin{tabular}{c|c||c|c}
T/\;K & EMK/\;V &T/\;K & EMK/\;V\\ 
\hline
160 & 0,2889 &   215&0,2860\\ 
165 & 0,2871& 220&0,2868 \\
170 & 0,2772 &  225&0,2877\\
175 & 0,2782& 230& 0,2885\\
180 &0,2792&235 &0,2893\\
185& 0,2803 &240&0,2900\\
190 &0,2814&245 &0,2903\\
195 &0,2826&250&0,2916\\
200 &0,2838&255 & 0,2923\\
205 & 0,2844&260& 0,2933\\
210& 0,2852&&\\
\end{tabular} 
\end{table}
\FloatBarrier
\subsection{$\Delta G_R$(T) gegen T}
Die Elektromotrische Kraft ist gleich dem Standardelektrodenpotential. 
\begin{equation}
c_{m,p} = \frac{UI\Delta t}{n\Delta T}
\end{equation}



\newpage


\subsection{Literaturverzeichnis}
1\quad Eckhold, Götz: \emph{Praktikum I zur Physikalischen Chemie}, Institut für Physikalische Chemie, Uni Göttingen, \textbf{2014}.
\vspace{0,5 cm}

2 \quad Eckhold, Götz: \emph{Statistische Thermodynamik}, Institut für Physikalische Chemie, Uni Göttingen, \textbf{2012}.

\vspace{0,5cm}

3 \quad Eckhold, Götz: \emph{Chemisches Gleichgewicht}, Institut für Physikalische Chemie, Uni Göttingen, \textbf{2015}.\\

\end{document}


	